\section{Architettura del progetto}

\immediate\write18{convert img/component-diagram.drawio.svg img/component-diagram.png}
\includegraphics[scale=0.6]{./img/component-diagram.png}


\paragraph{Descrizione classe:}
È una classe che rappresenta le coordinate nella scacchiera.
Quindi, l'insieme delle righe sarà così composto:
$ R := \{A,B,C,D,E,F,G,H\}$

Mentre l'insieme delle colonne:
$C := \{1,2,3,4,5,6,7,8\}$
Una coordinata non è altro che: $ R X C$.

\paragraph{Costruttori}
Sono presenti quattro costruttori:
\begin{enumerate}
\item Costruttore vuoto
\item Costruttore passando indici riga e colonna
\item Costruttore passando notazione stringa scacchistica
\item Costruttore tramite altra Coords
\end{enumerate}

\paragraph{Attributi:}
\begin{itemize}
  \item file: Colonna della scacchiera
  \item rank: Riga della scacchiera
\end{itemize}

\paragraph{Metodi:}
\begin{enumerate}
\item Operatori:
  \begin{enumerate}
    \item ==
    \item !=
    \item =
  \end{enumearte}
\item update, per modifica le coordiante
  \begin{enumerate}
    \item Passando un'altra coordianta
    \item Passando le esplicite coordinate riga e colonna
  \end{enumearte}
\item Metodi statici di controllo:
  \begin{enumerate}
    \item isValid: per l'indice
    \item isLetter: per la colonna
    \item isNumber: per la riga
    \item isInBounds: Se non eccede la scacchiera 8x8.
  \end{enumearte}
\end{enumerate}

\subsection{Board}
\paragraph{Descrizione classe:}
Rappresenta la foto di una scacchiera in questo momento.

\paragraph{Costruttori}
Sono presenti due costruttori:

\begin{enumerate}
  \item Costruttore vuoto: Genera la posizione classica degli scacchi.
  \item (std::string): Costruttore che prende un FEN
  costruendo la relativa posizione.
\end{enumerate}

\paragraph{Attributi:}
\begin{lstlisting}
  bitmap board
\end{lstlisting}

La bitmap contiene in totale 3 informazioni:
\begin{enumerate}
  \item Indice, che viene trasformato in coordinate
  \item Tipo del pezzo tramite 3 bit
  \item Colore del pezzo tramite 1 bit
\end{enumerate}

\immediate\write18{convert img/piecebit.drawio.svg img/piecebit.png}
\includegraphics[scale=0.6]{./img/piecebit.png}

La posizione \textit{board} 0 rappresenta la casella: a8.
Procedendo con l'indice di \textit{board}, iniziando dalla
casella \textit{a8} (casa in alto a sinistra nella scacchiera),
si procede da da sinistra verso destra e dell'alto verso il basso.

Riposto alcune corrispondenze per spiegare ulteriormente:
\begin{itemize}
  \item Indice 7: h8
  \item Indice 9: b7
  \item Indice 18: c6
  \item Indice 37: f4
\end{itemize}

\paragraph{Metodi:}

\subsection{Engine}
\paragraph{Descrizione classe:}
E' la classe che gestisce il gioco.
Media l'interazione tra l'utente e le componenti.

\paragraph{Costruttori}
E' presente un solo costruttore:

\begin{enumerate}
  \item Costruttore vuoto: serve a inizializzare le componenti interne.
\end{enumerate}

\paragraph{Attributi:}
\begin{lstlisting}
  Board board;
\end{lstlisting}

\paragraph{Metodi:}
