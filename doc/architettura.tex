\section{Architettura del progetto}

\subsection{Componenti generali}

\paragraph{Coords}
È una classe che rappresenta le coordinate nella scacchiera.
Quindi, l'insieme delle righe sarà così composto:
$ R := \{A,B,C,D,E,F,G,H\}$

Mentre l'insieme delle colonne:
$C := \{1,2,3,4,5,6,7,8\}$

Una coordinata non è altro che: $ R \bigtimes C$.


\subsection{Board}

\paragraph{Descrizione classe:}
Rappresenta la foto di una scacchiera in questo momento.

\paragraph{Costruttori}
Sono presenti due costruttori:

\begin{enumerate}
  \item Costruttore vuoto: Genera la posizione classica degli scacchi.
  \item (std::string): Costruttore che prende un FEN
  costruendo la relativa posizione.
\end{enumerate}

\paragraph{Attributi:}
\begin{lstlisting}
\item bitmap board
\end{lstlisting}

La bitmap contiene in totale 3 informazioni:
\begin{enumerate}
  \item Indice, che viene trasformato in coordinate
  \item Tipo del pezzo tramite 6 bit
  \item Colore del pezzo tramite 2 bit
\end{enumerate}

La posizione \textit{board} 0 rappresenta la casella: a8.
Procedendo con l'indice di \textit{board}, iniziando dalla
casella \textit{a8} (casa in alto a sinistra nella scacchiera),
si procede da da sinistra verso destra e dell'alto verso il basso.

Riposto alcune corrispondenze per spiegare ulteriormente:
\begin{itemize}
  \item Indice 7: h8
  \item Indice 9: b7
  \item Indice 18: c6
  \item Indice 37: f4
\end{itemize}

\paragraph{Metodi:}


\subsection{Engine}

\paragraph{Descrizione classe:}
E' la classe che gestisce il gioco.
Media l'interazione tra l'utente e le componenti.

\paragraph{Costruttori}
E' presente un solo costruttore:

\begin{enumerate}
  \item Costruttore vuoto: serve a inizializzare le componenti interne.
\end{enumerate}

\paragraph{Attributi:}
\begin{lstlisting}
  Board board;
  Menu menu;
\end{lstlisting}

\paragraph{Metodi:}
