\section{Architettura del progetto}

\subsection{Componenti generali}

\paragraph{Coords}
È una classe che rappresenta le coordinate nella scacchiera.
Quindi, l'insieme delle righe sarà così composto:
$ R := \{A,B,C,D,E,F,G,H\}$

Mentre l'insieme delle colonne:
$C := \{1,2,3,4,5,6,7,8\}$

Una coordinata non è altro che: $ R \bigtimes C$.


\subsection{Board}

\paragraph{Descrizione classe:}
Rappresenta la foto di una scacchiera in questo momento.
Per questa ragione, Board e' una classe contenente un:

\paragraph{Costruttori}
Sono presenti due costruttori:

\begin{enumerate}
  \item Costruttore vuoto: Genera la posizione classica degli scacchi.
  \item (std::string): Costruttore che prende un FEN
  costruendo la relativa posizione.
\end{enumerate}

\paragraph{Attributi:}
\begin{listing}
  std::array<Piece, 64> board;
\end{listing}

La posizione dell'array \textit{board} 0 rappresenta la casella: a8.
Procedendo con l'indice di \textit{board}, iniziando dalla
casella \textit{a8} (casa in alto a sinistra nella scacchiera),
si procede da da sinistra verso destra e dell'alto verso il basso.

Riposto alcune corrispondenze per spiegare ulteriormente:
\begin{itemize}
  \item Indice 7: h8
  \item Indice 9: b7
  \item Indice 18: c6
  \item Indice 37: f4
\end{itemize}

\paragraph{Metodi:}


