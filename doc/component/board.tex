\subsection{Board}
\paragraph{Descrizione classe:}
Rappresenta la foto di una scacchiera in questo momento.

\paragraph{Costruttori}
Sono presenti due costruttori:

\begin{enumerate}
  \item Costruttore vuoto: Genera la posizione classica degli scacchi.
  \item (std::string): Costruttore che prende un FEN
  costruendo la relativa posizione.
\end{enumerate}

\paragraph{Attributi:}
\begin{lstlisting}
  bitmap board
\end{lstlisting}

La bitmap contiene in totale 3 informazioni:
\begin{enumerate}
  \item Indice, che viene trasformato in coordinate
  \item Tipo del pezzo tramite 3 bit
  \item Colore del pezzo tramite 1 bit
\end{enumerate}

\immediate\write18{convert img/piecebit.drawio.svg img/piecebit.png}
\includegraphics[scale=0.6]{./img/piecebit.png}

La posizione \textit{board} 0 rappresenta la casella: a8.
Procedendo con l'indice di \textit{board}, iniziando dalla
casella \textit{a8} (casa in alto a sinistra nella scacchiera),
si procede da da sinistra verso destra e dell'alto verso il basso.

Riposto alcune corrispondenze per spiegare ulteriormente:
\begin{itemize}
  \item Indice 7: h8
  \item Indice 9: b7
  \item Indice 18: c6
  \item Indice 37: f4
\end{itemize}

\paragraph{Metodi:}
Ancora in fase di definizione.

\paragraph{Decisioni progettuali}
Sapendo che altrove nel progetto si creeranno diverse
istanze di \textit{Board} abbiamo cercato di renderla il
più possibile leggera a livello di memoria.
L'attuale decisione ultima è quella descritta della bit-board.
Questa soluzione permette di avere una board contenuta in \textit{32 byte}.
($4 bit * 64 caselle = 256 bit / 8 byte/bit = 32 byte$)
Il resto dei metodi ove possibile è statico per non avere peso sulla singola istanza.
