\subsection{Coords}
\paragraph{Descrizione classe:}
È una classe che rappresenta le coordinate nella scacchiera.
Quindi, l'insieme delle righe sarà così composto:
$ R := \{A,B,C,D,E,F,G,H\}$

Mentre l'insieme delle colonne:
$C := \{1,2,3,4,5,6,7,8\}$
Una coordinata non è altro che: $ R X C$.

\paragraph{Costruttori}
Sono presenti quattro costruttori:
\begin{enumerate}
\item Costruttore vuoto
\item Costruttore passando indici riga e colonna
\item Costruttore passando notazione stringa scacchistica
\item Costruttore tramite altra Coords
\end{enumerate}

\paragraph{Attributi:}
\begin{itemize}
  \item file: Colonna della scacchiera
  \item rank: Riga della scacchiera
\end{itemize}

\paragraph{Metodi:}
\begin{enumerate}
\item Operatori:
  \begin{enumerate}
    \item ==
    \item !=
    \item =
  \end{enumerate}
\item update, per modifica le coordiante
  \begin{enumerate}
    \item Passando un'altra coordianta
    \item Passando le esplicite coordinate riga e colonna
  \end{enumerate}
\item Metodi statici di controllo:
  \begin{enumerate}
    \item isValid: per l'indice
    \item isLetter: per la colonna
    \item isNumber: per la riga
    \item isInBounds: Se non eccede la scacchiera 8x8.
  \end{enumerate}
\end{enumerate}

\paragraph{Decisioni progettuali}
Da subito ci siamo accorti che le coordiante sarebbero state qualcosa
di importante.
Il primo approccio è stato quello di usare una \textit{struct}.
Tuttavia, dopo alcuni commit abbiamo preferito
passare a una \textit{class} avendo \textit{Coords} diversi
metodi propri e controlli da effettuare.
