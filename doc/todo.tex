\section{Lista delle cose da fare}

Presentiamo una lista di cose da fare per
ogni componente del progetto.
Nota: L'ordine conta! Trovi prima le cose più necessarie e in fondo quelle meno.

\subsection{Board}

\begin{enumerate}
\item Controllare se effettivamente il costruttore di Board funzioni.
\item Aggiungere controlli ed eccezioni per la costruttore
della posizione a parte dal fen.
\end{enumerate}

\subsection{Engine}

\begin{enumerate}
\item Aggiungere dei test per tutto.
\item Aggiungere stampa risultati e menu di continuazione.
\item Prendere il numero come string e poi parsarlo a uint8_t.
  Attualmente fa crashare il programma se in input si mette un -n.
\item Sistemare \textit{isMate}.
\item Sistemare \textit{saveGame}.
\item Sistemare \textit{loadGame}.
\item Sistemare \textit{takePlayerTurn}.
\item Sistemare \textit{takeEngineTurn}.
\item Sistemare \textit{playGameVsEngine}.
\item Aggiungi supporto multipli salvataggi
\end{enumerate}

\subsection{Coords}
\begin{enumerate}
\item Aggiungere dei test per tutto.
\end{enumerate}

\subsection{Printer}
\begin{enumerate}
\item Aggiungere dei test per tutto.
\item Aggiungere Menu per scelta colore.
\end{enumerate}

\subsection{Piece}
\begin{enumerate}
\item Aggiungere dei test per tutto.
\end{enumerate}

\subsection{Runner}
\begin{enumerate}
\item Modificare il file .bat
\item Creare il file .bash
\end{enumerate}

\subsection{Nuove funzionalità}

\begin{enumerate}
\item Implementare una versione di gioco con orologio
funzionante in tempo reale.

\item Creare un motore per giocare contro il computer offline.
\end{enumerate}
