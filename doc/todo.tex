\section{Lista delle cose da fare}

Presentiamo una lista di cose da fare per
ogni componente del progetto.
Nota: L'ordine conta! Trovi prima le cose più necessarie e in fondo quelle meno.

\subsection{Board}

\begin{enumerate}
\item Controllare se effettivamente il costruttore di Board funzioni.
\item Aggiungere controlli ed eccezioni per la costruttore
della posizione a parte dal fen.
\item Fare funzioni più piccole a partire da funzione da fen to board.
\end{enumerate}

\subsection{Engine}

\begin{enumerate}
\item Aggiungere dei test per tutto.
\item Aggiungere stampa risultati e menu di continuazione.
\item Prendere il numero come string e poi parsarlo a uint8_t.
  Attualmente fa crashare il programma se in input si mette un -n.
\item Sistemare \textit{isMate}.
\item Sistemare \textit{saveGame}.
\item Sistemare \textit{loadGame}.
\item Sistemare \textit{takePlayerTurn}.
\item Sistemare \textit{takeEngineTurn}.
\item Sistemare \textit{playGameVsEngine}.
\item Aggiungi supporto multipli salvataggi
\end{enumerate}

\subsection{Coords}
\begin{enumerate}
\item Aggiungere dei test per tutto.
\end{enumerate}

\subsection{Printer}
\begin{enumerate}
\item Aggiungere dei test per tutto.
\item Aggiungere Menu per scelta colore.
\end{enumerate}

\subsection{Piece}
\begin{enumerate}
\item Aggiungere dei test per tutto.
\end{enumerate}

\subsection{Runner}
\begin{enumerate}
\item Modificare il file .bat
\item Creare il file .bash
\end{enumerate}

\subsection{Nuove funzionalità}

\begin{enumerate}
\item Implementare una versione di gioco con orologio
funzionante in tempo reale.

\item Creare un motore per giocare contro il computer offline.
\end{enumerate}

\subsection{Idee per Evaluation}
  \subsubsection{Generale:}
    \begin{itemize}
      \item delta di materiale
      \item mobilita' pezzi
      \item centralita' pezzi
      \item king's safety
      \item pawn islands 
      \item pedoni doppi/tripli
      \item pedoni in prossimita' di promozione (7o e 2o rank)
      \item pinned pieces 
      \item numero mosse legali totali
      \item occupazione delle ali.
      \item triangolazione re per guadagnare tempo.
      \item dono greco.
      \item sacrificio pezzi minori per vantaggio sviluppo.
      \item forchetta
      \item alfieri in fianchetto
      \item attacchi di scoperta
      \item pezzi sospesi (+malus)
      \item pezzi difesi (+bonus)
      \item sacrificio semplificativo
      \item evitare il cambio nello svantaggio
      \item cambio forzante
      \item valore nel far perdere l'arrocco
      \item cavallo messo in avamposto non scacciabile
      \item scacco di scoperta
    \end{itemize}

  \subsubsection{Endgame (+middlegame)}
    \begin{itemize}
      \item mobilita' re
      \item pedoni passati
      \item pedoni passati supportati da altri pedoni
      \item Cattura pedoni centrali contro laterali.
      \item Mosse Zugzwang 
    \end{itemize}

\subsection{Idee per la search}
  \subsubsection{Generale}
    \begin{itemize}
      \item cercare prima e piu' a fondo se si da' scacco
      \item cercare prima ogni mossa che cattura un pezzo
      \item cercare piu' a fondo se si vince un pezzo
      \item cercare per ultimo ogni mossa che perde materiale
      \item cercare prima ogni mossa che centralizza un pezzo (o che si avvicina al re opposto)
    \end{itemize}

  \subsubsection{Middlegame}
    \begin{itemize}
      \item Cercare in profondità di attaccare le caselle deboli dell'avversario
    \end{itemize}
  
  \subsubsection{Endgame}
    \begin{itemize}
      \item cercare prima le mosse che diminuiscono la mobilita' del re avversario
      \item cercare prima le mosse che obbligano il re avversario ad avvicinarsi al bordo scacchiera
      \item cercare prima le promozioni di pedoni
    \end{itemize}